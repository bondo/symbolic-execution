\documentclass[11pt]{report}
%\documentclass[twoside,11pt,openright]{report}

\usepackage[utf8]{inputenc}
\usepackage[american]{babel}
\usepackage{a4}
\usepackage{latexsym}
\usepackage{amssymb}
\usepackage{amsmath}
\usepackage[T1]{fontenc}
\usepackage{lmodern}
\usepackage{color}
\usepackage{datetime}
\usepackage[numbers]{natbib}

\renewcommand*\ttdefault{txtt}

\newcommand{\todo}[1]{{\color[rgb]{.5,0,0}\textbf{$\blacktriangleright$#1$\blacktriangleleft$}}}

\begin{document}

%%%%%%%%%%%%%%%%%%%%%%%%%%%%%%%%%%%%%%%%%%%%%%%%%%%%%%%%%%%%%%%%%%%%%%%

\pagestyle{empty} 
\pagenumbering{roman} 
\vspace*{\fill}\noindent{\rule{\linewidth}{1mm}\\[4ex]
{\Huge\sf Symbolic execution of Python}\\[2ex]
{\huge\sf Bjarke Bondo Andersen, 20096300}\\[2ex]
\noindent\rule{\linewidth}{1mm}\\[4ex]
\noindent{\Large\sf Project work, Computer Science\\[1ex] 
\monthname\ \the\year  \\[1ex] Advisor: Anders Møller\\[15ex]}\\[\fill]}
\clearpage

%%%%%%%%%%%%%%%%%%%%%%%%%%%%%%%%%%%%%%%%%%%%%%%%%%%%%%%%%%%%%%%%%%%%%%%

\pagestyle{plain}
% \chapter*{Abstract}
% \addcontentsline{toc}{chapter}{Abstract}


\setcounter{tocdepth}{1}
\tableofcontents
\pagenumbering{arabic}
\setcounter{secnumdepth}{2}

%%%%%%%%%%%%%%%%%%%%%%%%%%%%%%%%%%%%%%%%%%%%%%%%%%%%%%%%%%%%%%%%%%%%%%%

\chapter{Introduction}
\label{ch:intro}

Symbolic execution can be thought of as being like random testing
(running programs with random values to see if it's well behaved, in
some sense), in that it's purpose is to discover program input that
triggers unwanted behaviour. Another way to think of symbolic
execution is as an alternative to static analysis that is
under-approximating (finding only real errors) instead of
over-approximating (finding all posible errors). \\

In it's original, purest form, as presented in \cite{King:1976},
symbolic execution is basically the idea that a function can be
executed with {\it symbolic} instead of {\it concrete}
values. 

Instead of a concrete program state, we have a symbolic state,
realized by keeping a map from program variables to symbolic values,
in terms of the input parameters (e.g. $\{\texttt{a} \mapsto
2\alpha-4, \texttt{b} \mapsto \beta\}$). Furthermore, a set of
symbolic assertions are maintained, extended at each branch -- when
going into the then branch, the branch condition is the assertion, and
when going into the else branch, the negation of the branch condition
is the assertion.

When the maintained assertions are conjugated, the result is called
the {\it path condition}, PC. The PC is a boolean expression,
constraining the input parameters to sets of concrete values that
would result in the same execution path. I.e. when the PC is
satisfiable, the solution is a map from input parameters to concrete
values that would follow exactly the same branching as the symbolic
execution. When a PC is not satisfiable or cannot be handled by the
constraint solver, that means we are unable to find an assignment of
concrete values to the input parameters that would result in the same
execution, and thus any further symbolic execution of the execution
path in question is useless. If both branches are feasible, fork the
symbolic execution and execute both. The PC is checked for
satisfiability every time it changes, so if an error state in the unit
under test is reached, the solution to the PC can be reported as a
concrete example that provokes that error state. \\

At the time it seemed hard to use the idea for any practical
analysis. For one thing, constraint solving technology was far behind
where we are today, making it both a lot slower and abel to deal only
with much simpler constraints. This imposed severe restrictions on the
allowed complexity of branching conditions, types of values, and
number of possible execution paths that could be explored in a
feasible time frame. Also, some constraints are simply unsolvable and
some parts of a concrete execution might be unavailable for symbolic
execution (e.g. RPC to a closed third-party server). \\

In \citeyear{DART} the paper \cite{DART} was published, which resulted
in a new serge of research in symbolic execution \cite{EXE,CUTE}. In
the paper, \citeauthor*{DART} shows that combining concrete and
symbolic execution can dramatically increase the overall efficiency by
substituting concrete for abstract values, thus simplifying hard or
unsolvable constraints. This means possibly sacrificing some
precision, but it solves a great many problems with the earlier
approach \cite{Cadar:2013}.


\section{Project}

Symbolic execution is typically targeted statically typed languages --
as far as the author knows \cite{Apollo}, which targets PHP, is
the only published exception. The dynamic nature of PHP is not much
explored in Apollo though, in the sense that program input is always
an HTTP request.

This, along with other language features and it's mainstraim
popularity, makes Python an intesting target language for symbolic
execution. \\

The project that this report documents has three goals:
\begin{itemize}
  \item To learn in general about symbolic execution.
  \item To find symbolic execution techniques applicable to Python.
  \item To find out if making source level transformations in order to
    automatically integrate the unit under test into a driver
    framework is a good strategy for implementing symbolic execution
    for Python.
\end{itemize}

%%%%%%%%%%%%%%%%%%%%%%%%%%%%%%%%%%%%%%%%%%%%%%%%%%%%%%%%%%%%%%%%%%%%%%%

\section{Overview}

As we suspected that making source level transformations might yield
the most bang for the buck, as opposed to building or modifying a
Python interpreter, to explore this idea became one of the project
goals.

This section will outline the implementation strategy on a high level,
followed by an joutline of the rest of the report.

\subsection{Implementation strategy}

When a Python function is chosen as the unit under test, it goes
through three phases:
\begin{description}
  \item[Normalization:] First the function is normalized,
    i.e.\ the code is transformed so that each statement and
    expression type is as simple and consistent as possible.
  \item[Instrumentation:] Then, the function is again transformed,
    this time inserting calls to a Python framework at every place
    where control needs to be transfered and information needs to be
    conveyed.
  \item[Execution:] The Python framework executes the chosen
    function. All the logic for doing concolic execution exists in
    this framework.
\end{enumerate}
The normalization and the instrumentation phases are implemented in
Haskell. We use the \verb|Language.Python| package to parse and
pretty-print Python code, and thus we only have to be concerned with
the AST.

We use Microsoft's SMT solver Z3 to solve constraints, because it has
a simple Python interface and can handle quite complex constraints.

\subsection{Outline}

In chapter \ref{ch:background}, we review some of the important,
relevant papers. Chapter \ref{ch:norm} describes the first
transformation phase, where the unit under test is normalized, so as
to reduce the number of cases to handle later on. Chapter
\ref{ch:instrumentation} describes the second transformation phase,
where calls to our Python framework are inserted. The Python framework
is described in chapter \ref{ch:symbolic}, while some of the things
that would have been in the framework, had there been more time, are
listed in chapter \ref{ch:future}. Chapter \ref{ch:conclusion}
concludes.

%%%%%%%%%%%%%%%%%%%%%%%%%%%%%%%%%%%%%%%%%%%%%%%%%%%%%%%%%%%%%%%%%%%%%%%

\chapter{Background}
\label{ch:background}

This chapter will review two papers that are relevant to the project.

The first paper, \cite{DART}, was the one that rebooted the interest
in symbolic execution by mixing in concrete execution, resulting in
what came to be known as {\it concolic} execution after the term was
coined in \cite{CUTE}.

The second paper, \cite{Apollo}, is as far as the author knows the the
only published tool for symbolic execution of a dynamically typed
language, PHP.

\section{\citet{DART}}

The paper presents the tool DART ({\it Directed Automated Random
  Testing}), which was a major breakthrough in the practical
usefulness of symbolic execution techniques. The primary innovation
was to automatically generate random concrete values satisfying a
given PC, and use these concrete values to simultaneously have a
concrete and symbolic state. This allowed the execution to continue
when the constraints went outside the constraint theory.

DART targets C, and basically works as follows: First, all input
parameters are initialized to random values. Then statements are
executed one at a time, each time executing the statement twice --
once symbolically and once concretely. If the statement in question is
an \verb|if| statement, there are two cases:
\begin{enumerate}
  \item If the condition is integer linear, calculate it's concrete
    value. Then, if the condition is true, add the symbolic value of
    the condition to the PC. Otherwise, add the negation of the
    symbolic value of the condition to the PC. Also, register that the
    new PC has been seen before.
  \item If the condition is {\it not} integer linear, simply follow
    the concrete value of the condition and leave the PC alone.
\end{enumerate}
If sone error occurs, or the program simply terminates, report the
error and find the next path condition to pursue. That is done by
removing the latest assertion as long as both that and it's negation
have been seen. If the PC is now empty, there are no more paths to
explore. Otherwise negating the last assertion, and feed the resulting
PC to the constraint solver. Continue this process untill the PC is
solvable, then restart the program with the PC solution as the input
parameters.

In the end, the lesson from DART is that when symbolic values become
to hard to work with, we can simply fall back on the concrete values.

\section{\citet{Apollo}}

This paper is farily recent (\citeyear{Apollo}), and thus somewhat
more sophisticated techniques are used. The accompanying tool as
called Apollo, and is build as a modification to an existing PHP
interpreter.

The basic idea is the same, i.e.\ Apollo is a concolic execution
tool. Because the target is PHP, used as a HTML generation tool, the
program input is a sequence HTTP request, representing a session of
user interaction, and the result for each request is a HTML
document. This presents a unique challenge -- the input HTTP requests
has to be build to reflect possible cases w.r.t.\ the intermediate
HTML results. This means the HTML has to be parsed, and so it only
natural require valid documents. Invalid documents are reported as
program errors, along with the input sequence that resulted in the
offending document. 

Many errors states can be reached through several execution paths, so
in order to minimize duplication and remove ``noise'' from the output,
a path constraint minimization algorithm is used. The algorithm
systimatically bulks together similar errors. When a bulk is found
where the intersection of the contained PCs is regarded as that bulks
PC, and if that PC still results in the same error, the bulk is kept
and the PC is minimized.

%%%%%%%%%%%%%%%%%%%%%%%%%%%%%%%%%%%%%%%%%%%%%%%%%%%%%%%%%%%%%%%%%%%%%%%

\chapter{Normalization}
\label{ch:norm}

We do a normalization in which statements and expressions are
transformed into \emph{atomic actions}. For example, expressions are
transformed into a single action on local variables and a list of
statements that assign the value of the subexpressions into
corresponding variables.

Example: \verb!a + b * 4! is transformed into the following list of
preceding statements
\begin{verbatim}
tmp_1 = 4
tmp_2 = b * tmp_1
\end{verbatim}
and the expression \verb!a + tmp_2!

\section{Supported language}
For now, only a subset of Python 3 is supported. This section will
list supported/unsupported expression and statement AST nodes.

\subsection{Supported expressions}
\begin{description}
  \item[Var:] Variables
  \item[Int, LongInt, Float, Imaginary, Bool, Tuple, List and Set:]
    Literals of the obvious kind.
  \item[None:] The \verb|None| literal.
  \item[Ellipsis:] The \verb|Ellipsis| literal.
  \item[ByteStrings, Strings and UnicodeStrings:] String literals.
  \item[Call:] Function calls and method invocations. Only ``normal''
    parameter passing supported, no named arguments or unpacking of
    lists and dictionaries. After normalization all arguments are
    variables.
  \item[BinaryOp:] The application of a binary operator. Chained
    comparisons are not supported yet. After normalization both sides
    of the BinaryOp are variables.
  \item[UnaryOp:] The application of a unary operator. After
    normalization the expression in the UnaryOp is a variable.
  \item[Paren:] Parenthesized expressions.
\end{description}

\subsection{Unsupported expressions}
\begin{description}
  \item[Subscript:] Expressions like \verb|x[y]|.
  \item[SlicedExpr:] Expressions like \verb|w[x:y:z]|.
  \item[CondExpr:] \verb|if|-\verb|then|-\verb|else| expressions.
  \item[Lambda:] Lambda expressions.
  \item[Yield:] The \verb|Yield| operator.
  \item[Generator:] Generators.
  \item[Dictionary:] Dictionary literals.
  \item[ListComp, DictComp and SetComp:] List, dictionary and set
    comprehensions.
  \item[Starred:] Expressions prefixed by a \verb|*|.
\end{description}

\subsection{Supported statements}
\begin{description}
  \item[Import and ImportFrom:] Import statements.
  \item[While:] While loops. Only supported when the condition is a
    variable.
  \item[For:] For loops. Only supported when the target is a single
    variable. After normalization the generator expression is also a
    variable.
  \item[Fun:] Function declarations. Only ``normal'' parameters,
    possibly with default values, are supported, no var-args or
    keyword dictionaries.
  \item[Class:] Class declarations.
  \item[Conditional:] \verb|if|-\verb|elif|-\verb|else| chains.
  \item[Assign:] Normal assignment. Only a single left hand side is
    supported. After normalization the right hand side is a literal, a
    function call or the application of a binary or unary operator.
  \item[AugmentedAssign:] Assignments like \verb|a += b|. After
    normalization this becomes an Assign node.
  \item[Return, Pass, Break, Continue, Global and NonLocal:]
    Statements of the same name. After normalization either nothing or
    a variable is returned.
  \item[Assert:] An assert statement. Only a single condition is
    supported. After normalization the argument is a variable.
  \item[StmtExpr:] Statements consisting of an expression. The
    expression is normalized as any other.
\end{description}

\subsection{Unsupported statements}
\begin{description}
  \item[Decorated:] Decorated statements, i.e. statements prefixed by
    \verb|@something|.
  \item[Try:] \verb|try|-\verb|except|-\verb|finally| statements.
  \item[Raise:] Raise an exception.
  \item[With:] Automated resource handling.
  \item[Delete:] Delete something, like the delete operator in
    ECMAScript.
  \item[Exec:] Compile and execute code on the fly, analogous to eval
    in ECMAScript.
\end{description}

\section{Normalizing {\tt if-elif-else}}
Assume we have a function $$simplifyToVariable: Expression \rightarrow
[Statement] \times Variable$$ that transforms an arbitrary expression
into a list of normalized statements, and a variable containing the
final value of the expression.

If we now have that $(stmts, var) = simplifyToVariable(t1)$, we can
normalize an \verb!if!-\verb!elif! chain
\begin{verbatim}
if t1:
  S1
elif t2:
  S2
elif t3:
  S3
else
  S4
\end{verbatim}
by first transforming it into a single \verb!if!-statement with the rest of the
\verb!if!-\verb!elif! chain in the \verb!else! clause
\begin{verbatim}
stmts
if var:
  S1
else
  if t2:
    S2
  elif t3:
    S3
  else
    S4
\end{verbatim}
and then recursing into the two branches.

\section{Normalizing $\lambda$-expressions}
All $\lambda$-expressions should be transformed into named (local)
functions. The body of a $\lambda$-expression is an expression, but
when normalized it may require a number of preceding statements. Thus
one cannot use a $\lambda$-expression as the result of a normalized
$\lambda$-expression. $\lambda$-expressions are not currently
supported, but should be fairly easy.

\section{Normalizing {\tt while}}
We only have support for {\tt while} where the condition is a single
variable. To support arbitrary conditions we would have to precede the
whole {\tt while} statement and all matching inner {\tt continue}
statements with the list of statements calculating the condition.

\section{Semantic deviation}
\paragraph{The short-circuit semantics of boolean operators are ignored;}
  instead expressions are always fully evaluated. This could be fixed
  at the cost of some output complexity, e.g. \verb|if a or b: S| should
  be normalized to
\begin{verbatim}
tmp_1 = a
tmp_2 = not tmp_1
if tmp_2: tmp_1 = b
if tmp_1: S
\end{verbatim}

%%%%%%%%%%%%%%%%%%%%%%%%%%%%%%%%%%%%%%%%%%%%%%%%%%%%%%%%%%%%%%%%%%%%%%%

\chapter{Instrumentation}
\label{ch:instrumentation}

We instrument the python code by inserting calls to our Python
framework into the source.

Example: The program
\begin{verbatim}
def foo(x):
    return 2*x
\end{verbatim}
is first normalized to
\begin{verbatim}
def foo(x):
    tmp_1 = 2
    tmp_2 = tmp_1 * x
    return tmp_2
\end{verbatim}
and then after inserting calls to our instrumentation code it becomes
\begin{verbatim}
def foo(x):
    symbolic_scope('x')
    symbolic_assign_literal('tmp_1', 2)
    tmp_1 = 2
    symbolic_assign_binop('tmp_2', '*', 'tmp_1', 'x')
    tmp_2 = tmp_1 * x
    symbolic_return('tmp_2')
    return tmp_2
\end{verbatim}

\section{Transformations}
Statements are transformed like this:

\begin{description}
  \item[Functions:] A call to {\tt symbolic\_scope} is inserted as the
    first statement of the function body.
  \item[{\tt while} loops:] An {\tt assert} is prepended to the body.
  \item[Conditionals:] In every {\tt then} branch an {\tt assert}
    statement is prepended. In every {\tt else} branch a {\tt refute}
    statement is prepended.
  \item[Assignment] We distinguish three kinds of assignment: literal,
    call and binop, resulting in calls to {\tt
      symbolic\_assign\_literal}, {\tt symbolic\_assign\_call} and
    {\tt symbolic\_assign\_binop} respectively. A literal assignment
    is when the right-hand-side is a literal value, e.g. a number or a
    string constant. A call assignment is an assignment where the
    right-hand-side is a function call. A binop assignment is an
    assignment where the right-hand-side is a binary operation on two
    variables.
  \item[{\tt return} statements:] Precede the statement with a call to
    {\tt symbolic\_return}. If a variable is returned (after
    normalization either a variable or nothing is returned), pass the
    name of that variable as a string to {\tt symbolic\_return}.
  \item[{\tt pass}, {\tt break} and {\tt continue} statements:] Leave
    them be.
\end{description}

\section{Supported language}
Compared to the normalization phase, only a tiny language is
supported. Not all of what the normalization can output is supported
in this phase.

By assumption, only statements needs to have accompanying
instrumentation calls, and so this section only talks about supported
and unsupported statements. Expressions on unsupported value types
should be handled by the Python framework.

\subsection{Supported statements}
\begin{description}
  \item[While:] The condition must be a single variable and there must
    be no else branch. Supporting an else branch should be
    straightforward.
  \item[Fun:] Parameters must be names -- no default values and no
    annotations. No annotations on the function itself either.
  \item[Conditional:] \verb|if| statements with a variable as
    condition, no \verb|elif| and possibly an \verb|else| branch. Note
    that this is exactly what the normalization outputs.
  \item[Assign:] Normal assignment, with a single variable on the left
    hand side. The right hand side should be a literal, a function
    call or a binary operation on two variables. This is consistent
    with the output of the normalization phase.
  \item[Return:] Return a variable or nothing. Consistent with
    normalization.
  \item[Pass, Break and Continue:] No instrumentation needed.
  \item[StmtExpr:] Only supported when the expression is a function
    call.
\end{description}

\subsection{Unsupported statements}
\begin{description}
  \item[Import and FromImport:] It probably would not demand any
    instrumentation, so support should be straightforward.
  \item[For:] It is not clear how the generator should be handled
    symbolically.
  \item[Class:] To handle classes well, we should be able to
    instantiate their fields with primitive values, new instances or
    old instances. We should then model their instances symbolically
    and pass them as input values to the function under
    test. \cite{Pasareanu:2010}
  \item[AugmentedAssign:] Removed by normalization, so no need to
    support.
  \item[Decorated, Try, Raise, With and Delete:] Not supported in
    normalization either.
  \item[Global and NonLocal:] Support for these will especially be
    demanding on the Python side of things, where these statements
    will complicate handling of scopes.
  \item[Assert:] Should be straightforward, and useful, to implement.
  \item[Exec:] Will likely be hard to implement with this
    implementation strategy -- would have been straightforward in a
    modified interpreter.
\end{description}

%%%%%%%%%%%%%%%%%%%%%%%%%%%%%%%%%%%%%%%%%%%%%%%%%%%%%%%%%%%%%%%%%%%%%%%

\chapter{Symbolic execution}
\label{ch:symbolic}

The Python framework for the actual symbolic execution is actually
fairly simple, mainly because of the extremely simple Python interface
of the constraint solver, Z3, which actually represents values as ASTs
of symbolic values.

We then inject functions (\verb|symbolic_scope|, \verb|symbolic_assert|,
etc.) into \texttt{builtins}, so they can be called from the client
code without us having to change the imports.

And finally we can invoke the unit under test, and wait for it to
finish or throw an exception (which is the only kind of errors we
recognize).

\section{Parameter types}
For now, the framework only supports integers. But if more types were
to be added, how should the types of the input parameters be chosen?
Several possibilities springs to mind:
\begin{itemize}
  \item Have the user specify the types
  \item Try every combination of types
  \item Add constraints describing the type of inputs when code where
    the type matters is run.
\end{itemize}
The last of these possibilities seems to be the most in the spirit of
symbolic execution, but the first one will likely be sufficient for
many situations, specially if a list of types could be specified for
each parameter so combinations of these could be brute forced.

As far as the author can tell, this is new ground, so some exploration
before setteling on something could well be beneficial.

\section{Example}

Given that \verb|~/symbolic-execution/transformer/tests/pythagoras.py| contains
\begin{verbatim}
def foo(a, b, c):
    if a*a + b*b == c*c and a > 0:
        sum = a+b+c
        if 0 < sum and sum < 15 and a < b and b < c:
            res = sum / (c - b - a + 2)
\end{verbatim}

the current state of our work yields the following output

\begin{verbatim}
~/symbolic-execution/instrumentation$ python3 instrument.py
Transform /home/bondo/symbolic-execution/transformer/tests/pythagoras.py.
Transformation successful.

No error occurs when ('a', 'b', 'c') = (0, 0, 0).
Path just executed:  (('tmp_8', False),)
Next execution path: (('tmp_8', True),)
Corresponding solver: [0 == 0, Not(Not(And(a*a + b*b == c*c, a > 0)))]
Constraints are satisfiable
Model: [c = -5, b = -4, a = 3, 0 = 0] 

No error occurs when ('a', 'b', 'c') = (3, -4, -5).
Path just executed:  (('tmp_8', True), ('tmp_18', False))
Next execution path: (('tmp_8', True), ('tmp_18', True))
Corresponding solver: [0 == 0,
 15 == 15,
 And(Not(Not(And(And(And(0 < a + b + c, a + b + c < 15),
                     a < b),
                 b < c))),
     And(a*a + b*b == c*c, a > 0))]
Constraints are satisfiable
Model: [c = 5, b = 4, a = 3, 15 = 15, 0 = 0] 

When ('a', 'b', 'c') = (3, 4, 5) a ZeroDivisionError exception is
        raised: division by zero
Traceback:
  /home/bondo/symbolic-execution/transformer/tests/pythagoras.out.py
        line 54 in function foo:
    res = sum / tmp_22
Path just executed:  (('tmp_8', True), ('tmp_18', True))
No more execution paths found
\end{verbatim}

Notice that constraints like \verb|15 == 15| are added to the solver
-- this is actually a variable on the left and an integer on the
right. When assigning an integer constant as a symbolic value, we
instead assign the variable with that integer as the name. This allows
us to completely rely on Z3's symbolic representation of values. This
might change when we introduce symbolic type coercion and primitive
types other than integers and booleans.

\section{Handling path conditions}
We store executed paths as a list of tuples containing \\
\verb|(symbolic_boolean_condition, variable_name, assert/refute)|.

Notice that because of the normalization, an execution path is
uniquely determined by the reduced list of \verb|(name, is_assert)|
tuples.

We find the next path to explore by this algorithm:
\begin{verbatim}
def find_next_path():
    if empty(): return None
    flip_last_condition()
    if seen_before():
        remove_last_condition()
        return find_next_path()
    return path()
\end{verbatim}

%%%%%%%%%%%%%%%%%%%%%%%%%%%%%%%%%%%%%%%%%%%%%%%%%%%%%%%%%%%%%%%%%%%%%%%

\chapter{Future work}
\label{ch:future}

This is list of code related things that the author never got around
to:

\begin{itemize}
  \item Bound execution of client code, either with a timeout or by
    the number of symbolic assertions made.
  \item Z3 can theoretically go into an infinite loop (under which
    conditions?), so set a timeout.
  \item Implement function calls in the execution, both for local and
    foreign (library or external) functions.
  \item Fall back on concrete values when Z3 fails.
  \item It would be nice to add constraints that force division by
    zero and other such errors when possible.
  \item Add instrumentation support for more types. First bool and
    floating point, then complex numbers, then tuples, then arrays,
    strings and arbitrary classes.
  \item Generally add support for more of Python.
  \item Add support for more types and experiment with ways to
    determain with witch types to initialize input parameters.
  \item Run tests to ensure the transformations are semantics
    preserving.
\end{itemize}

%%%%%%%%%%%%%%%%%%%%%%%%%%%%%%%%%%%%%%%%%%%%%%%%%%%%%%%%%%%%%%%%%%%%%%%

\chapter{Conclusion}
\label{ch:conclusion}

Symbolic execution has come a long way in recent years, and seems to
finally have the potential to help find a lot of bugs in real world
applications. But although many new tricks have been found since it
sprang back into life, it still feels like there is much to explore.

In the complexity and flexibility of Python, some features of the
language seems fairly hard to implement with the current strategy of
source code transformations. Maybe extending an existing interpreter
would be a shorter path to full Python support.

%%%%%%%%%%%%%%%%%%%%%%%%%%%%%%%%%%%%%%%%%%%%%%%%%%%%%%%%%%%%%%%%%%%%%%%

\bibliographystyle{plainnat}
\bibliography{report}

\end{document}
